\section{Synopsis}
\label{sec:synopsis}





{\tiny Dette projekt omhandler en vurdering af trygheden af trafikken i Nytorv/Østerågade området i Aalborg. Der er set på overvejelser omkring fodgængernes tryghed og sikkerhed i området.
\\
Der er blevet redegjort for områdets beliggenhed og struktur, og om begrebet Shared Space. Her er forklaret, hvilke årsager der kan være medvirke til at tiltrække mange trafikanter.
\\
Områdets udseende er blevet diskuteret med henblik på Shared Space. Der er herunder set på sammenligninger og folks færden i området.
\\
Der er lavet interviews af fodgængere, hvor der er set på deres opfattelse af trygheden i området. Herunder er de blevet udarbejdet for at få en forståelse af nogle generelle holdninger og trafikale problemer.
\\
Der er foretaget nogle observationer ved et fodgængerfelt i området, hvor fodgængernes sikkerhed over for cyklisterne er diskuteret. Der er herunder foretaget beregninger af TA-værdien og lavet adfærdsregistrering.
\\
Trafiktællinger er blevet foretaget, og der er udarbejdet flow kort over området. Herunder er der lavet beregninger af ÅDT.
\\
Forslag til løsninger med henblik på de foretaget observationer, interviews og trafiktællinger er blevet diskuteret og områdets udseende er blevet perspektiveret.}
