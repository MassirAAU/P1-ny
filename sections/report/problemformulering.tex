%mainfile: master.tex
\chapter{Problemformulering}
\label{chap:problemformulering}

%Nytorv/Østerågade er i centrum af Aalborg by. Der er mange moderne butikker, kaffebarer, restauranter og Aalborg`s hovedbibliotek, som giver bosiddende og besøgende et sted at nyde deres fritid. Der er derfor mange mennesker som går ture, shopper, drikker kaffe og spiser sammen med venner i området på Nytorv/Østerågade. Aalborg centrum er et hyggeligt sted at være og med meget leben. Ifølge indledningen kan projektet se, at Aalborgs befolkning stiger. Nytorv/Østerågade er blevet trafiktalt og det har vist sig, at trafikmængden stiger år efter år. Det drejer sig om både busser, cyklister og fodgængere. Der bliver trafikkonflikter med hensyn til shared space.
~\\\\
Det er interresant at se på, hvordan trafikken fungerer nede ved Nytorv/Østerågade området, i og med at der færdes så mange trafikanter, og er et af de centrale steder i Aalborg by. De mange shoppingfaciliteter og cafémuligheder gør det til et attraktivt sted, især for fodgængere, at færde sig i, og derfor er det især denne trafikantgruppe, der er i højsædet i dette projekt. Problemstillingerne er måske ikke så store den dag i dag, men set i perspektivet af, at Aalborg har befolkningsvækst, kan der i fremtiden opstå store trafikale problemer i området. Det er derfor relavent at undersøge, hvilke konflikter, forbigående fodgængere og cyklister mener er til stede mellem dem og de andre trafikantgrupper.
Det er også interresant at se på om området fungerer som et Shared Space område eller noget helt andet.
Det leder efterfølgende til en masse ideer om en eller flere eventuelle løsninger på de trafikale problemer, der kan tænkes at være til stede i området, og dermed løse fremtidens trafikale problemstillinger. Det er især et fokus på en effektivisering af trafik flowet, der i dette projekt cirkuleres omkring. Udgangspunktet i projektet vil blive taget i følgende spørgsmål.
\\\\
Hvilke konflikter finder sted mellem cyklister og fodgængere på Nytorv/Østerågade?
\\
Hvordan kan et løsningsforslag løse de eventuelle konflikter?
\\
Hvordan kan der eventuelt skabes et bedre trafik flow ved området?
\\\\

\section{Afgrænsning}
\label{sec:afgraensning}

I dette projekt er der fokuseret på trafikkonflikter mellem cyklister og fodgængere på Nytorv/Østerågade i Aalborg. Fokuspunktet er fodgængernes tryghed overfor cyklisterne og bilerne. Der vil altså ikke blive fokuseret på buschauførrenes syn på sikkerheden eller dem der kører privatbiler, da de desuden er udelukket fra området. Der er blevet valgt i projektet at afgrænse sig til at lave kvalitative interviews af fodgængere og observationer af Nytorv. Desuden er der en afgrænsning i forhold til de forskellige trafikantgrupper, da der vil blive lavet interviews, udelukkende med forbipasserende som går eller cykler på Nytorv. Der vil blive undersøgt, om fodgængerne føler sig trygge, når de går på Nytorv/Østerågade i Aalborg. Der tages udgangspunkt i konflikter der kan forekomme mellem cyklister og fodgængere, og hvordan projektet i området kan skabe et bedre trafik flow.

I undersøgelsen har projektet som formål at vurdere disse trafikkonflikter, og ved hjælp af de forskellige observationsmetoder og interviews, komme med et eller flere forslag til nogle konstruktive løsningsforslag som kan forøge trygheden for cyklister samt fodgængere ved Nytorv/Østerågade området.
