\section{Hvorfor færdes der mange mennesker på Nytorv?}
\label{sec:hvorfor_faerdes_der_mange_mennesker_paa_nytorv?}
Det er interresant at se på hvorfor der færdes så mange mennesker på Nytorv/Østerågade, som der gør, for at få en forståelse af, hvorfor der til tider kan være meget trafik dernede, og derved for nogle en give en utryghed i at færdes dernede på gåben.
Aalborg er med sine 109.092 indbyggere i 2014 Danmarks 4. største by \autocide{bystorelse}.%(http://www.denstoredanske.dk/Danmarks_geografi_og_historie/Danmarks_geografi/Jylland/Jylland_-_byer/Aalborg Aalborg kommune) 
Det er især på Nytorv/Østerågade, at der færdes mange mennesker, som enten transportere sig med bus, på cykel eller er gående. Den store tiltrækning på Nytorv ligger i dels de mange forskellige offentlige og kommercielle servicefunktioner, og dels de mange shopping- og cafémuligheder \autocite{attraktiv}. 
%(http://www.aalborgkommuneplan.dk/kommuneplanrammer/midtbyen/aalborg-midtby/default.aspx) 
Et af projekterne er Aalborgs havnefront, som kan ses på \cref{fig:midtby} side \pageref{fig:midtby} ved Honørkajen, der i flere år har været under omdannelse. Således er havnen gået fra at være industri arbejdsplads, til en af byens attraktioner. Havnefronten tilbyder arkitektur, gastronomi og gode faciliteter til at samles og hygge med venner.
%(http://www.visitaalborg.dk/aalborg/aalborgs-havnefront) 
Shoppingmuligheder er der som sagt også mange af. Her finder man de 2 store gågader, Bispensgade og Algade, hvor forskellige mode- og specialforretninger ligger side om side. Gågaderne er adskilt af Østerågade, hvorfra Nytorv også ligger, som tilbyder andre shoppingmuligheder med Salling, Friis Citycenter og Føtex \autocide{shopping}.
%( http://www.visitdenmark.dk/da/nordjylland/shopping/shopping-i-aalborg) 
Nytorv og Østerågade ligger derfor meget centreret i Aalborg, og er et stort samlingspunkt for beboerne i Aalborg og omegn. Østerågade ligger i forlængelse af Boulevarden og strækker sig til Toldbod Pl., som er forbundet med strandvejen, hvor havnefronten ligger. På grund af de mange shoppingmuligheder og de fælles samlingspunkter, kan der på nogle tidspunkter af dagen være rigtig mange trafikanter, som færdes nede på Nytorv. Ligeledes færdes der også mange trafikanter, som cykler eller pendler til og fra arbejde hver dag, som på nogle tidspunkter af dagen også kan skabe et højt trafikflow. Derfor er det interresant at se på, om fodgængerne på Nytorv føler sig trygge, når der er mange cykler, busser og for den sags skyld også biler, som færdes dernede. For Østerågade og Nytorv er ikke kun et knudepunkt for fodgængere. Området bliver også brugt som busholdeplads for de mange by- og metrobusser som kører igennem hver time. Hele 32 busser med forskellige destinationer har stoppested i området \autocide{busser}. Det er til og med blevet gjort ulovligt for privat billister at køre i området.  På Nytorv ser man en fællesbelægning, både på fortov og vej, hvilket kan ses på figur \cref{fig:nytorv} side \pageref{fig:nytorv}. Der er heller ikke meget afmærkning, og bliver benyttet af trafikanterne som et slags fællesareal. Man kunne måske argumentere for, at Nytorv bliver brugt som et slags Shared Space, som vil blive beskrevet i det følgende. 