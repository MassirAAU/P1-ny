\chapter{Diskussion}
\label{chap:diskussion}
For at kunne afgøre, hvad eller hvilke eventuelle løsningsforslag, som er de bedste, er det væsentligt at se på fordele og ulemper ved de forskellige forslag. Ved at anlægge en rundkørsel skabes der et naturligt bedre trafikflow og antallet af kollisioner trafikantgrupperne i mellem formindskes, da alle trafikantgrupper er tvunget til at øge deres trafikale fokus som konsekvens af rundkørslen. Én af ulemperne ved at anlægge en rundkørsel vil på sigt blive en ringere offentlig transport i området, da det vil forsinke bustrafikken markant at føre dem ind gennem en rundkørsel, fyldt med cyklister og fodgængere. En rundkørsel er særdeles effektiv at bruge som løsningsforslag et sted, hvor der ofte sker kollisioner i mellem biler og cykler eller biler og fodgængere. Eftersom det største trafikaktuelle problemfokus på Nytorv ikke er bilerne, men derimod mere cyklerne, ville en rundkørsel muligvis gøre mere skade end gavn, hvis man valgte at anlægge en sådan én.
Hvis vi i stedet retter fokus på løsningsforslaget med at anlægge en busgrav/sætte hejse/sænke pæle op, er dette forslag ligesom det tidligere også fokuseret på bilproblemet omkring Nytorv. Det ville afhjælpe problemet med de mange uvedkommende biler, der hver eneste dag kører ind på området, trods påbud på adskillige skilte, som tilsyneladende ikke har den ønskede virkning. En af fordelene bliver hermed, at Nytorv bliver helt fri for uvedkommende biler, og dette er medvirkende til at skabe et mere forudsigeligt og dermed forventeligt og måske mere trygt trafikmiljø. Flere steder i landet har man prøvet at anlægge busgrave med stor succes, men det ses også, at der sommetider kører biler over busgravene. Dette medfører en forsinkelse på busserne, og er umiddelbart den eneste ulempe.
Det sidste løsningsforslag vi har valgt at prøve og komme med, handler om at anlægge en cykelbane i hele området i og omkring Nytorv. Fordelene er, at dette er med til at øge fokus på cyklisterne, og det giver et mere klart billede af, hvor cyklerne skal være på vejen. En af de overvejende ulemper er, at en cykelbane vil skabe gnidninger busserne, cyklisterne og busserne i mellem, når busserne holder ind ved stoppestederne, da Nytorv er et trafik knudepunkt for busser i Aalborg, og er stedet folk i Aalborg går hen for at søge en bus. Da der i forvejen er et forholdsvis smalt fortov nogle steder nede ved Nytorv, bliver det ligeså svært at få plads til de eventuelle cykelbaner. De to første løsningsforslag har fokus på eventuelle bilproblemer, hvorimod det sidste løsningsforslag har til formål at afhjælpe cykelproblemer i trafikken. Eftersom det største trafikale problem nede ved Nytorv er mellem cyklister og fodgængere, er det oplagt at videreudvikle på ideen omkring en cykelbane.
I rapporten har den helt centrale problemstilling været, at undersøge trafikale problemer mellem fodgængere og cyklister ved fodgængerfeltet. Gennem de beregnede TA-værdier og de mange interviews, er der blevet vurderet, at der er et trafikalt problem netop på dette sted mellem cyklisterne og fodgængerne. Bilerne og busserne er gennem de foretagede interviews ikke blevet vurderet til at være et stort problem, og derfor er løsningsforslaget om cykelbanen valgt at blive diskuteret her, som den bedste løsning, for netop denne undersøgelse.
En etablering af en cykelbane vil have til formål at lede cyklisten inde for en afmærket sti, og på den måde helliggøre og fastholde cyklisten inden for afmærkningen. På den måde formås der at kontrollere cyklisternes kørebane og opretholde cyklistens fokus i selve kørebanen. Det formodes, at cyklisten vil få øje på en fodgængeren hurtigere, da der ikke er andre trafikanter cyklisten nu skal holde øje med. Cykelbanens helliggørelse skærper heraf cyklistens fokus fra hele området til kørebanen og dermed fodgængeren. Det kan muligvis være med til, at man vil kunne observere mange flere tidlige samspil end sene samspil, mellem de to trafikanter, og herved øge trafiksikkerheden. Et sådant formål, ville en etablering af en rundkørsel eller bussluse ikke opfylde. De tre løsningsforslag er relativt simple, og det er også derfor, at de er blevet valgt ud i projektet. Da løsningerne kun er antagede og ikke underbygget, er det klart, at en cykelbane ikke nødvendigvis vil løse nogle af de konflikter, der finder sted. Men det er klart, at det vil kunne guide cyklisterne, og have et bidrag til et mere sikkert trafikmiljø på Nytorv/Østerågade området.
